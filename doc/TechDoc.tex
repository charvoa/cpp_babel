\documentclass{article}

\usepackage{listings}
\usepackage[english]{babel}
\usepackage{color}

\interlinepenalty 10000

\definecolor{dkgreen}{rgb}{0,0.6,0}
\definecolor{gray}{rgb}{0.5,0.5,0.5}
\definecolor{mauve}{rgb}{0.58,0,0.82}

\lstset{frame=tb,
  language=Java,
  aboveskip=3mm,
  belowskip=3mm,
  showstringspaces=false,
  columns=flexible,
  basicstyle={\small\ttfamily},
  numbers=none,
  numberstyle=\tiny\color{gray},
  keywordstyle=\color{blue},
  commentstyle=\color{dkgreen},
  stringstyle=\color{mauve},
  breaklines=true,
  breakatwhitespace=true,
  tabsize=3
}

\title{Spyke (Babel Project)}
\author{La Pintade}
\date{5 Octobre 2015 - 8 Novembre 2015}

\begin{document}

  \maketitle
  \tableofcontents


  \newpage
  \section{About}
    This document is a technical documentation about Spyke.
    Skype is a software developped by the La Pintade team, based in Nice.
  \newpage

  \section{GUI}
  In this section we will see the main classes used in order to display the GUI of Spyke.

  \begin{lstlisting}
    int		main(int ac, char **av)
    {
      MyApplication app(ac, av);
      LoginWidget *login;

      login = new LoginWidget;

      QFile File2("./gui/stylesheetLogin.qss");
      File2.open(QFile::ReadOnly);
      QString StyleSheet2 = QLatin1String(File2.readAll());

      /* Applying it */
      login->setStyleSheet(StyleSheet2);

      [...]

      /* Show login window */
      login->setAttribute(Qt::WA_DeleteOnClose);
      login->show();

      return app.exec();
    }
  \end{lstlisting}

  First we load stylsheets for each widgets. And we display the login widget.

  \begin{lstlisting}
    void LoginWidget::checkLogin()
    {
      MainWidget *widget = new MainWidget();
      QString user = _editUsername->text();
      QString pass = _editPassword->text();

      _userString = user.toUtf8().constData();
      _passString = pass.toUtf8().constData();

      g_PTUser.logUser(*this, &LoginWidget::validateLogin);

      widget->setAttribute(Qt::WA_DeleteOnClose);
      if (_login)
        {
          widget->show();
          deleteLater();
        }
    }

  \end{lstlisting}

  The user has to log himself in the system in order to use Spyke. 

\end{document}
